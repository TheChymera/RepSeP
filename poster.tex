\documentclass[xcolor=table,english]{beamer}
\usepackage[orientation=portrait,size=a0]{beamerposter}
\mode<presentation>{\usetheme{ZH}}
\usepackage[utf8]{inputenc}
\usepackage{ragged2e}
\usepackage{verbatimbox}
\usepackage[font=small,justification=justified]{caption}
\usepackage{array}
\usepackage{tabularx}

\usepackage[british]{babel} % decent hyphenation, avoiding e.g. anal-ysis
\usepackage[iso]{isodate}
\usepackage{sansmath}
\usepackage{booktabs}
\usepackage{graphicx}
\usepackage{graphviz}
\usepackage{makecell}
\usepackage{minted}
\usepackage{multicol}
\usepackage{siunitx}
\usepackage{subcaption}
\usepackage[section]{placeins}

% Needs to be loaded after hyperref
\usepackage{cleveref}

% PythonTeX
\usepackage[autoprint=false, gobble=auto, keeptemps=all, pyfuture=all]{pythontex} % create figures on-line directly from python!
\usepackage{pgf}
\begin{pythontexcustomcode}[begin]{py}
import os, sys
almost_this_path = os.path.abspath(os.path.dirname(__file__))
this_path_base, _ = os.path.split(almost_this_path)
this_path = os.path.join(this_path_base,"pythontex")

import matplotlib
import matplotlib.pyplot as plt
from pylab import gcf

# Set the prefix used for figure labels
fig_label_prefix = 'fig'
# Track figure numbers to create unique auto-generated names
fig_count = 0

def pytex_tab(script, caption='', label='', options=''):
    import sys
    try:
        from StringIO import StringIO
    except ImportError:
        from io import StringIO
    import contextlib

    @contextlib.contextmanager
    def stdoutIO(stdout=None):
        old = sys.stdout
        if stdout is None:
            stdout = StringIO()
        sys.stdout = stdout
        yield stdout
        sys.stdout = old

    with stdoutIO() as s:
        exec(open(script).read(), locals())

    tab = latex_table(s.getvalue(), caption=caption, label=label, options=options)
    return tab

def pytex_fig(script, conf=[], caption='', label='', multicol=False):
    '''
    Executes a python script while applying the custom style.

    Notes
    -----
    We go about this in a somewhat roundabout fashion - always applying `DOC_STYLE`, and then additionally applying a context style which may be only `[DOC_STYLE]` in case no other config file is specified.
    This is because scripts need to be executed in a while statement lest rcParams become persistent between figures.
    Conversely, the text engine part of the configuration needs to be applied outside of the context statement, because it will not work inside it.
    '''
    pytex.add_dependencies(script)
    try:
        document_style = DOC_STYLE
    except NameError:
        document_style = 'pythontex/minimal.conf'
    plt.style.use(document_style)
    try:
        if isinstance(conf, basestring):
            conf = [conf]
    except NameError:
        if isinstance(conf, str):
            conf = [conf]
    figure_styles = [document_style]+conf
    pytex.add_dependencies(*figure_styles)
    with plt.style.context(figure_styles):
        exec(open(script).read())
    if multicol:
        fig = latex_figure(save_fig(), "figure*", caption=caption, label=label)
    else:
        fig = latex_figure(save_fig(), "figure", caption=caption, label=label)
    return fig

def figure_by_path(figure_path,textheight_frac=1,caption=None,label=None):
    latex_code = "\\begin{figure}\n"
    latex_code += "\\centering\\includegraphics[width={textheight_frac}\\textheight]{{{figure_path}}}\n".format(textheight_frac=textheight_frac,figure_path=figure_path)
    latex_code += "\\vspace{-.5em}\n"
    latex_code += "\\caption{{{caption}}}\n".format(caption=caption)
    latex_code += "\\label{{fig:{label}}}\n".format(label=label)
    latex_code += "\\end{figure}\n"
    return latex_code

def save_fig(name='', legend=False, fig=None, ext='.pgf'):
    '''
    Save the current figure (or `fig`) to file using `plt.save_fig()`.
    If called with no arguments, automatically generate a unique filename.
    Return the filename.
    '''
    # Get name (without extension) and extension
    if not name:
        global fig_count
        # Need underscores or other delimiters between `input_*` variables
        # to ensure uniqueness
        name = 'auto_fig_{}-{}'.format(pytex.id, fig_count)
        fig_count += 1
    else:
        if len(name) > 4 and name[:-4] in ['.pgf', '.svg', '.png', '.jpg']:
            name, ext = name.rsplit('.', 1)

    # Get current figure if figure isn't specified
    if not fig:
        fig = gcf()
    fig.savefig(name + ext)
    fig.clear()
    plt.cla()
    plt.clf()
    plt.close()
    plt.close('all')
    return name

def latex_environment(name, content='', option=''):
    '''
    Simple helper function to write the `\begin...\end` LaTeX block.
    '''
    return '\\begin{%s}%s\n%s\n\\end{%s}' % (name, option, content, name)

def latex_table(table, caption='', label='', options=''):
    content = table
    content += "\\caption{%s\\label{%s:%s}}\n" % (caption, "tab", label)
    return latex_environment("table", content=content, option=options)

def latex_figure(name, figure, caption='', label='', width=1):
    ''''
    Auto wrap `name` in a LaTeX figure environment.
    Width is a fraction of `\textwidth`.
    '''
    if not name:
        name = save_fig()
    content = '\\centering\n'
    content += '\\makeatletter\\let\\input@path\\Ginput@path\\makeatother\n'
    content += '\\input{%s.pgf}\n' % name
    if not label:
        label = name
    if caption and not caption.rstrip().endswith('.'):
        caption += '.'
    if caption:
        # `\label` needs to be in `\caption` to avoid issues in some cases
        content += "\\caption{%s\\label{%s:%s}}\n" % (caption, fig_label_prefix, label)
    return latex_environment(figure, content, '[htp]')

pytex.bio_fignum = 0
#global pytex # try without this line
def bio_fig(gdd, fname=None, caption=None, label=None):
#        global pytex # and this one, should work
        if fname is None:
            fname = 'pythontex-files-pres/biopython_fig_{0}-{1}.pdf'.format(pytex.id, pytex.bio_fignum)
        gdd.write(fname, "PDF")
        template = '''
    \\begin{{figure}}
    \\centering
    \\includegraphics{{{fname}}}
    \\caption{{ {label} {caption} }}
    \\end{{figure}}
    '''
        if caption is None:
            caption = ''
        if label is None:
            label = ''
        else:
            if not label.startswith('fig:'):
                label = 'fig:' + label
            label = '\\label{{{0}}}'.format(label)
        template = template.format(fname=fname.rsplit('.', 1)[0], label=label, caption=caption)
        print(template)
        pytex.add_created(fname)
        pytex.bio_fignum += 1
        return template
\end{pythontexcustomcode}
\begin{pythontexcustomcode}[end]{py}
\end{pythontexcustomcode}

\begin{pythontexcustomcode}[begin]{py}
pytex.add_dependencies(
	'lib/utils.py',
	'lib/categorical.py',
	'data/JogB.tsv'
	)
\end{pythontexcustomcode}
% Single-session PythonTeX codeblocks
\newcounter{pysessioncounter}
\newcommand{\sessionpy}{%
          \edef\sessionpysession{session\arabic{pysessioncounter}}%
            \stepcounter{pysessioncounter}%
              \expandafter\py\expandafter[\sessionpysession]}

% SIunitx customizations detect-all will use the current font for typesetting
\sisetup{per-mode=symbol, detect-all, range-units = single}
\newcommand\SIci[5]{\SI{#1}{#2}, {#3}CI: \SIrange{#4}{#5}{#2}}

\begin{pythontexcustomcode}[begin]{py}
DOC_STYLE="poster/main.conf"
pytex.add_dependencies(
	DOC_STYLE,
	'poster/wide.conf',
	'poster/3dplot.conf',
	'poster/1col.conf',
	)
\end{pythontexcustomcode}

\newcolumntype{Z}{>{\centering\arraybackslash}X} % centered tabularx columns


\title{\href{https://github.com/TheChymera/RepSeP}{\Large github.com/TheChymera/RepSeP}\\\vspace{.15em}Reproducible Publishing --- Reference Poster Implementation}
% \title{Longitudinal opto-pharmaco-fMRI of Selective\\ Serotonin Reuptake Inhibition}
\author{Horea-Ioan Ioanas$^{1}$, Markus Rudin$^{1}$}
\institute[ETH]{$^{1}$Institute for Biomedical Engineering, ETH and University of Zurich}
\date{\today}

\newlength{\columnheight}
\setlength{\columnheight}{0.881\textheight}

\begin{document}

\begin{myverbbox}[\small]{\tabcode}
\begin{table}[]
      \py{
            pytex_tab(
                  script='scripts/stim_table.py', label='sp',
                  caption='BIDS \cite{bids} event file table, from \cite{irsabi}',
                  options_pre='\\centering \\resizebox{0.9\\textwidth}{!}{',
                  data='data/JogB.tsv', options_post='}',
                  )
          }
\end{table}
\end{myverbbox}
\begin{myverbbox}[\small]{\spcode}
\py{pytex_fig(
        'scripts/3dplot.py',
        conf='poster/3dplot.conf', label='3dplot',
        caption='A 3D plot. Plot script from matplotlib exampes \cite{matplotlib}.',
        )}
\end{myverbbox}

\begin{frame}
\begin{columns}
	\begin{column}{.42\textwidth}
		\begin{beamercolorbox}[center]{postercolumn}
			\begin{minipage}{.98\textwidth}  % tweaks the width, makes a new \textwidth
				\parbox[t][\columnheight]{\textwidth}{ % must be some better way to set the the height, width and textwidth simultaneously
					\begin{myblock}{Abstract}
						As scientific work becomes increasingly collaborative and automated, reproducibility becomes increasingly vital for sharing and integrating scientific results.
For most researchers, however, reproducibility remains a nebulous ideal, the benefits of which are considered more theoretical and indirect, than practical and immediate.
Here we showcase an open-source reference implementation of a technology stack which makes the benefits of reproducibility accessible via a reusable document template.

\vspace{.8em}

The prevalent and currently most accessible medium of exchange for high-level (i.e. semantic) scientific results is that of the document.
As this medium (including e.g. posters and articles) is static, it encourages the creation of work which is unreproducible.
We present an infrastructure which addresses this issue, without compromising content sharing standards, by automatically generating variable article elements (e.g. figures and statistics) directly from code and data.

					\end{myblock}\vfill
					\vspace{-0.3em}
					\begin{myblock}{Dependency Management}
						\begin{figure}
							\captionsetup{width=.9\linewidth}
							\vspace{-1.2em}
							\includedot[width=0.9\textwidth]{data/dependencies}
							\vspace{-1.8em}
							\caption{
								Small excerpt and conceptual representation of software dependencies for research articles doing even only simple neuroimaging analysis.
								First-level dependencies are highlighted in green.
							}
							\vspace{1em}
						\end{figure}

						Explicit, unambiguous declaration (e.g. conforming to the Package Manager Specification \cite{pms}) of first-level dependencies is vital for software environment reproducibility.
						Given such a specification, an effective package manager can automatically resolve all downstream steps (typically extending into the hundreds or thousands of packages) \cite{ng}.

					\end{myblock}\vfill
					\vspace{-0.3em}
					\begin{myblock}{Technologies}
						\vspace{0.75em}
						\begin{minipage}{.28\textwidth}
							\begin{figure}
								\includegraphics[width=0.9\textwidth]{img/pythontex}
							\end{figure}
						\end{minipage}~
						\begin{minipage}{.68\textwidth}
							Arbitrarily complex plots leveraging the full capacity of matplotlib \cite{matplotlib}, higher-level packages building on it, or other Python plotting packages, are generated live on document compilation.
							\vspace{0.6em}

							The core technology of this infrastructure is provided by Python\TeX{} \cite{pytex}, which comes complete with \textit{codeblock} dependency tracking (n.b. not to be confused with dependency management for the software itself).
							All figures are generated on the initial execution of a \LaTeX{} compile script, and subsequently regenerated when either code, data, or styling dependencies are changed.
						\end{minipage}
					\end{myblock}\vfill
					\vspace{-0.3em}
					\begin{myblock}{3D Plots}
						\vspace{0.6em}
						\py{pytex_fig(
							'scripts/3dplot.py',
							conf='poster/3dplot.conf', label='3dplot', caption='A 3D plot, from the matplotlib exampes \cite{matplotlib}.',
							)}
						\vspace{1.5em}
						The above image is dynamically generated on document compilation by inserting the following code in the \TeX{} document source:
						\vspace{.5em}
                       		                \begin{figure}
                                                        \fcolorbox{tlg}{elg}{\spcode}
                                                \end{figure}
					\end{myblock}\vfill
					\vspace{-0.3em}
					\begin{myblock}{Outlook}
						\begin{itemize}
						        \item We are looking for \textbf{testers}, to apply this reference implementation to their own work (we have used it to great satisfaction in numerous articles, including \cite{irsabi}).
							\item We are looking for potential \textbf{co-authors} for the reference article implementation, which is to be submitted to an academic journal.
							\item We are looking for potential \textbf{web developers} or \textbf{co-founders} to launch a platform offering RepSeP-based publishing services (yes, a journal of reproducible code-based articles!).
						\end{itemize}
					\end{myblock}\vfill
		}\end{minipage}\end{beamercolorbox}
	\end{column}
	\begin{column}{.59\textwidth}
		\begin{beamercolorbox}[center]{postercolumn}
			\begin{minipage}{.98\textwidth} % tweaks the width, makes a new \textwidth
				\parbox[t][\columnheight]{\textwidth}{ % must be some better way to set the the height, width and textwidth simultaneously
					\begin{myblock}{Why do we need this?}
						\vspace{0.5em}
						\begin{center}
							\begin{minipage}{.47\textwidth}
							Why do we need code-based publishing?
								\begin{itemize}
									\item Transparency $\longrightarrow$ verifiability
									\item Reproducibility $\longrightarrow$ hackability, reusability
									\item Version management $\longrightarrow$ sustainability
								\end{itemize}
							\end{minipage}
							\begin{minipage}{.48\textwidth}
							Why do we need distributed publishing?
								\begin{itemize}
									\item No external entry barrier $\longrightarrow$ citizen science
									\item No institutional bias $\longrightarrow$ free science
									\item \textit{Less} publication bias $\longrightarrow$ honest science
								\end{itemize}
							\end{minipage}
						\end{center}
					\end{myblock}\vfill
					\begin{myblock}{Split Violin Plots}
						\vspace{0.1em}
						\py{pytex_subfigs(
							[
								{'script':'scripts/vc_violin.py',
									'label':'vcv',
									'conf':'poster/1col.conf',
									'options_pre':'{.48\\textwidth}',
									},
						      {'script':'scripts/vcc_violin.py',
									'label':'vccv',
									'conf':'poster/1col.conf',
									'options_pre':'{.48\\textwidth}',
									},
								],
							environment='figure',
							options_pre='\captionsetup{width=.9\linewidth}',
							caption='
								Violin plots highlighting both the distribution densities and quartiles in a multifactorial comparison (from \cite{irsabi}).
								The style is adapted in the source code of this document to improve quartile styling in excess of the capabilities offered by the upstream package, seaborn \cite{seaborn}.
								',
							label='fig:vc',
							)}
						\vspace{0.5em}
						Custom plotting options can also be used, by distributing customization code inside the article source.
						For \cref{fig:vc}, a patched module from the original code is distributed in \colorbox{elg}{\texttt{lib/categorical.py}}, and preferentially imported in the script files (e.g. \colorbox{elg}{\texttt{scripts/violin.py}}).
					\end{myblock}\vfill
					\begin{myblock}{Statistics and Tables}
						\vspace{0.3em}
						\begin{minipage}{.51\textwidth}
							Automatically computed and formatted inline statistics:
							\begin{itemize}
								\item \py{pytex_printonly('scripts/anova.py')}
								\item Processing Factor: \py{boilerplate.fstatistic('Processing', condensed=True)}
								\item Template Factor: \py{boilerplate.fstatistic('Template', condensed=True)}
								\item Processing:Template Intearction: \py{boilerplate.fstatistic('Processing:Template', condensed=True)}
							\end{itemize}
						\end{minipage}
						\begin{minipage}{.46\textwidth}
							\begin{table}[]
								\vspace{0.4em}
									\py{
										pytex_tab(
											script='scripts/stim_table.py',
											label='sp',
											caption='BIDS \cite{bids} event file table, from \cite{irsabi}',
											options_pre='\\centering \\resizebox{0.9\\textwidth}{!}{',
											data='data/JogB.tsv',
											options_post='}',
											)
										}
								\vspace{0.4em}
							\end{table}
						\end{minipage}
						\vspace{-1em}

						Text elements can also be auto-generated from code and data, allowing inline statistics to be dynamic.
						Such elements can be based on single scripts (e.g. \colorbox{elg}{\texttt{\textbackslash py\{pytex\_printonly('scripts/anova.py')\}}}), or parameterized script calls (e.g. \colorbox{elg}{\texttt{\textbackslash py\{boilerplate.fstatistic('Processing', condensed=True)\}}}), allowing the same model to be used and different factors to be reported in different locations.
						\vspace{.7em}

						\begin{minipage}{.7\textwidth}
							\begin{figure}
								\fcolorbox{tlg}{elg}{\tabcode}
							\end{figure}
						\end{minipage}\hfill
						\begin{minipage}{.25\textwidth}
							Additionally, scripts such as the one invoked in the code block to the left allow tab or comma separated value files to be automatically read and typeset as \LaTeX{} tables.
						\end{minipage}
					\end{myblock}\vfill
					\begin{myblock}{Manual Anchors}
						\py{pytex_fig(
							'scripts/bsc_percentage.py', conf='poster/wide.conf', label='bsc_percentage',
							caption='
								Percentage of Bachelor’s degrees conferred to women in the U.S.A. by major (1970-2011).
								Plot script from matplotlib exampes \cite{matplotlib}.
								',
							)}
						\vspace{0.5em}
						The style application via hierarchical matplotlib configuration files (global, per-document, per-script --- in ascending order of priority) allows the selfsame script results to be adapted to individual document types.
						Multiple views of the same data analysis summary (e.g. a plot) can thus rely on the same code, avoiding divergent editing.
						Even sensitive plot elements, such as anchors, remain stable throughout various style applications, as exemplified here.
					\end{myblock}\vfill
					\begin{myblock}{References}
                                                \vspace{-1em}
                                                \begin{multicols}{2}
                                                        \scriptsize
                                                        \bibliographystyle{ieeetr}
							\bibliography{common/bib}
                                                \end{multicols}
                                                \vspace{-1em}
					\end{myblock}\vfill
		}\end{minipage}\end{beamercolorbox}
	\end{column}
\end{columns}
\end{frame}
\end{document}
