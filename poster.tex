\documentclass[xcolor=table,english]{beamer}
\usepackage[orientation=portrait,size=a0]{beamerposter}
\mode<presentation>{\usetheme{ZH}}
\usepackage[utf8]{inputenc}
\usepackage{ragged2e}
\usepackage{verbatimbox}
\usepackage[font=small,justification=justified]{caption}
\usepackage{array}
\usepackage{tabularx}

\usepackage[british]{babel} % decent hyphenation, avoiding e.g. anal-ysis
\usepackage[iso]{isodate}
\usepackage{sansmath}
\usepackage{booktabs}
\usepackage{graphicx}
\usepackage{sty/graphviz}
\usepackage{makecell}
\usepackage{minted}
\usepackage{multicol}
\usepackage{siunitx}
\usepackage{subcaption}
\usepackage[section]{placeins}

% Needs to be loaded after hyperref
\usepackage{cleveref}

% PythonTeX
\usepackage[autoprint=false, gobble=auto, keeptemps=all, pyfuture=all]{pythontex} % create figures on-line directly from python!
\usepackage{pgf}
% The import package is needed for relative figure referencing in PGFs stored under the pythontex job directory
\usepackage{import}
\begin{pythontexcustomcode}[begin]{py}
pytex.add_dependencies(
	'lib/utils.py',
	'lib/categorical.py',
	'data/JogB.tsv'
	)
\end{pythontexcustomcode}
% Single-session PythonTeX codeblocks
\newcounter{pysessioncounter}
\newcommand{\sessionpy}{%
          \edef\sessionpysession{session\arabic{pysessioncounter}}%
            \stepcounter{pysessioncounter}%
              \expandafter\py\expandafter[\sessionpysession]}

% SIunitx customizations detect-all will use the current font for typesetting
\sisetup{per-mode=symbol, detect-all, range-units = single}
\newcommand\SIci[5]{\SI{#1}{#2}, {#3}CI: \SIrange{#4}{#5}{#2}}

% Fix for matplotlib PGF wonkiness which isn't interpreted correctly by pdflatex
\DeclareUnicodeCharacter{2212}{-}

\begin{pythontexcustomcode}[begin]{py}
DOC_STYLE="poster/main.conf"
JOBNAME="poster"
pytex.add_dependencies(
	DOC_STYLE,
	'poster/wide.conf',
	'poster/3dplot.conf',
	'poster/1col.conf',
	)
\end{pythontexcustomcode}

% This does not go into the common header as it needs to be imputed after the document-specific variables are defined.
\input{/usr/share/repsep/functions.py}

\newcolumntype{Z}{>{\centering\arraybackslash}X} % centered tabularx columns


\title{\href{https://github.com/TheChymera/RepSeP}{\Large github.com/TheChymera/RepSeP}\\\vspace{.15em}Reproducible Publishing --- Reference Poster Implementation}
% \title{Longitudinal opto-pharmaco-fMRI of Selective\\ Serotonin Reuptake Inhibition}
\author{Horea-Ioan Ioanas$^{1}$, Markus Rudin$^{1}$}
\institute[ETH]{$^{1}$Institute for Biomedical Engineering, ETH and University of Zurich}
\date{\today}

\newlength{\columnheight}
\setlength{\columnheight}{0.881\textheight}

\begin{document}

\begin{myverbbox}[\small]{\tabcode}
\begin{table}[]
      \py{
            pytex_tab(
                  script='scripts/stim_table.py', label='sp',
                  caption='BIDS \cite{bids} event file table, from \cite{irsabi}',
                  options_pre='\\centering \\resizebox{0.9\\textwidth}{!}{',
                  data='data/JogB.tsv', options_post='}',
                  )
          }
\end{table}
\end{myverbbox}
\begin{myverbbox}[\small]{\spcode}
\py{pytex_fig(
        'scripts/3dplot.py',
        conf='poster/3dplot.conf', label='3dplot',
        caption='A 3D plot. Plot script from matplotlib exampes \cite{matplotlib}.',
        )}
\end{myverbbox}

\begin{frame}
\begin{columns}
	\begin{column}{.42\textwidth}
		\begin{beamercolorbox}[center]{postercolumn}
			\begin{minipage}{.98\textwidth}  % tweaks the width, makes a new \textwidth
				\parbox[t][\columnheight]{\textwidth}{ % must be some better way to set the the height, width and textwidth simultaneously
					\begin{myblock}{Abstract}
						As the scientific world becomes increasingly large and increasingly reliant on automation, reproducibility becomes more and more vital for the successful development of both the scientific community and research technologies.
For most researchers, however, reproducibility remains a nebulous ideal, the benefits of which are considered more theoretical and indirect, than practical and immediate.
In this article we share a reference implementation of an article, which makes the benefits of reproducibility accessible and reusable.
In this article we put forth an explicit reproducibility model, which is both simple and consistently applicable throughout the scientific process, and thus to the work of any scientist.
We proceed to select a set of technologies which strongly facilitate reproducibility, and present to the reader a plethora of reproductions of peer-reviewed scientific work.
While this framework is as all encompassing as possible, we see it thus as an excellent reference implementation which which is reusable across a wide range of topics.

					\end{myblock}\vfill
					\vspace{-0.3em}
					\begin{myblock}{Dependency Management}
						\begin{figure}
							\captionsetup{width=.9\linewidth}
							\vspace{-1.2em}
							\includedot[width=0.9\textwidth]{data/dependencies}
							\vspace{-1.8em}
							\caption{
								Small excerpt and conceptual representation of software dependencies for research articles doing even only simple neuroimaging analysis.
								First-level dependencies are highlighted in green.
							}
							\vspace{1em}
						\end{figure}

						Explicit, unambiguous declaration (e.g. conforming to the Package Manager Specification \cite{pms}) of first-level dependencies is vital for software environment reproducibility.
						Given such a specification, an effective package manager can automatically resolve all downstream steps (typically extending into the hundreds or thousands of packages) \cite{ng}.

					\end{myblock}\vfill
					\vspace{-0.3em}
					\begin{myblock}{Technologies}
						\vspace{0.75em}
						\begin{minipage}{.28\textwidth}
							\begin{figure}
								\includegraphics[width=0.9\textwidth]{img/pythontex}
							\end{figure}
						\end{minipage}~
						\begin{minipage}{.68\textwidth}
							Arbitrarily complex plots leveraging the full capacity of matplotlib \cite{matplotlib}, higher-level packages building on it, or other Python plotting packages, are generated live on document compilation.
							\vspace{0.6em}

							The core technology of this infrastructure is provided by Python\TeX{} \cite{pytex}, which comes complete with \textit{codeblock} dependency tracking (n.b. not to be confused with dependency management for the software itself).
							All figures are generated on the initial execution of a \LaTeX{} compile script, and subsequently regenerated when either code, data, or styling dependencies are changed.
						\end{minipage}
					\end{myblock}\vfill
					\vspace{-0.3em}
					\begin{myblock}{3D Plots}
						\vspace{0.6em}
						\py{pytex_fig(
							'scripts/3dplot.py',
							conf='poster/3dplot.conf', label='3dplot', caption='A 3D plot, from the matplotlib exampes \cite{matplotlib}.',
							)}
						\vspace{1.5em}
						The above image is dynamically generated on document compilation by inserting the following code in the \TeX{} document source:
						\vspace{.5em}
                       		                \begin{figure}
                                                        \fcolorbox{tlg}{elg}{\spcode}
                                                \end{figure}
					\end{myblock}\vfill
					\vspace{-0.3em}
					\begin{myblock}{Outlook}
						\begin{itemize}
						        \item We are looking for \textbf{testers}, to apply this reference implementation to their own work (we have used it to great satisfaction in numerous articles, including \cite{irsabi}).
							\item We are looking for potential \textbf{co-authors} for the reference article implementation, which is to be submitted to an academic journal.
							\item We are looking for potential \textbf{web developers} or \textbf{co-founders} to launch a platform offering RepSeP-based publishing services (yes, a journal of reproducible code-based articles!).
						\end{itemize}
					\end{myblock}\vfill
		}\end{minipage}\end{beamercolorbox}
	\end{column}
	\begin{column}{.59\textwidth}
		\begin{beamercolorbox}[center]{postercolumn}
			\begin{minipage}{.98\textwidth} % tweaks the width, makes a new \textwidth
				\parbox[t][\columnheight]{\textwidth}{ % must be some better way to set the the height, width and textwidth simultaneously
					\begin{myblock}{Why do we need this?}
						\vspace{0.5em}
						\begin{center}
							\begin{minipage}{.47\textwidth}
							Why do we need code-based publishing?
								\begin{itemize}
									\item Transparency $\longrightarrow$ verifiability
									\item Reproducibility $\longrightarrow$ hackability, reusability
									\item Version management $\longrightarrow$ sustainability
								\end{itemize}
							\end{minipage}
							\begin{minipage}{.48\textwidth}
							Why do we need distributed publishing?
								\begin{itemize}
									\item No external entry barrier $\longrightarrow$ citizen science
									\item No institutional bias $\longrightarrow$ free science
									\item \textit{Less} publication bias $\longrightarrow$ honest science
								\end{itemize}
							\end{minipage}
						\end{center}
					\end{myblock}\vfill
					\begin{myblock}{Split Violin Plots}
						\vspace{0.1em}
						\py{pytex_subfigs(
							[
								{'script':'scripts/vc_violin.py',
									'label':'vcv',
									'conf':'poster/1col.conf',
									'options_pre':'{.48\\textwidth}',
									},
						      {'script':'scripts/vcc_violin.py',
									'label':'vccv',
									'conf':'poster/1col.conf',
									'options_pre':'{.48\\textwidth}',
									},
								],
							environment='figure',
							options_pre='\captionsetup{width=.9\linewidth}',
							caption='
								Violin plots highlighting both the distribution densities and quartiles in a multifactorial comparison (from \cite{irsabi}).
								The style is adapted in the source code of this document to improve quartile styling in excess of the capabilities offered by the upstream package, seaborn \cite{seaborn}.
								',
							label='fig:vc',
							)}
						\vspace{0.5em}
						Custom plotting options can also be used, by distributing customization code inside the article source.
						For \cref{fig:vc}, a patched module from the original code is distributed in \colorbox{elg}{\texttt{lib/categorical.py}}, and preferentially imported in the script files (e.g. \colorbox{elg}{\texttt{scripts/violin.py}}).
					\end{myblock}\vfill
					\begin{myblock}{Statistics and Tables}
						\vspace{0.3em}
						\begin{minipage}{.51\textwidth}
							Automatically computed and formatted inline statistics:
							\begin{itemize}
								\item \py{pytex_printonly('scripts/anova.py')}
								\item Processing Factor: \py{boilerplate.fstatistic('Processing', condensed=True)}
								\item Template Factor: \py{boilerplate.fstatistic('Template', condensed=True)}
								\item Processing:Template Intearction: \py{boilerplate.fstatistic('Processing:Template', condensed=True)}
							\end{itemize}
						\end{minipage}
						\begin{minipage}{.46\textwidth}
							\begin{table}[]
								\vspace{0.4em}
									\py{
										pytex_tab(
											script='scripts/stim_table.py',
											label='sp',
											caption='BIDS \cite{bids} event file table, from \cite{irsabi}',
											options_pre='\\centering \\resizebox{0.9\\textwidth}{!}{',
											data='data/JogB.tsv',
											options_post='}',
											)
										}
								\vspace{0.4em}
							\end{table}
						\end{minipage}
						\vspace{-1em}

						Text elements can also be auto-generated from code and data, allowing inline statistics to be dynamic.
						Such elements can be based on single scripts (e.g. \colorbox{elg}{\texttt{\textbackslash py\{pytex\_printonly('scripts/anova.py')\}}}), or parameterized script calls (e.g. \colorbox{elg}{\texttt{\textbackslash py\{boilerplate.fstatistic('Processing', condensed=True)\}}}), allowing the same model to be used and different factors to be reported in different locations.
						\vspace{.7em}

						\begin{minipage}{.7\textwidth}
							\begin{figure}
								\fcolorbox{tlg}{elg}{\tabcode}
							\end{figure}
						\end{minipage}\hfill
						\begin{minipage}{.25\textwidth}
							Additionally, scripts such as the one invoked in the code block to the left allow tab or comma separated value files to be automatically read and typeset as \LaTeX{} tables.
						\end{minipage}
					\end{myblock}\vfill
					\begin{myblock}{Manual Anchors}
						\py{pytex_fig(
							'scripts/bsc_percentage.py', conf='poster/wide.conf', label='bsc_percentage',
							caption='
								Percentage of Bachelor’s degrees conferred to women in the U.S.A. by major (1970-2011).
								Plot script from matplotlib exampes \cite{matplotlib}.
								',
							)}
						\vspace{0.5em}
						The style application via hierarchical matplotlib configuration files (global, per-document, per-script --- in ascending order of priority) allows the selfsame script results to be adapted to individual document types.
						Multiple views of the same data analysis summary (e.g. a plot) can thus rely on the same code, avoiding divergent editing.
						Even sensitive plot elements, such as anchors, remain stable throughout various style applications, as exemplified here.
					\end{myblock}\vfill
					\begin{myblock}{References}
                                                \vspace{-1em}
                                                \begin{multicols}{2}
                                                        \scriptsize
                                                        \bibliographystyle{ieeetr}
							\bibliography{common/bib}
                                                \end{multicols}
                                                \vspace{-1em}
					\end{myblock}\vfill
		}\end{minipage}\end{beamercolorbox}
	\end{column}
\end{columns}
\end{frame}
\end{document}
